
\section{Challenges Faced}
Initially, when I started this project, it began as something that I wanted to implement in C++. I thought C++ would be the best language to take this through with since it naturally made sense to me choose C++ because of my previous experience with DSA in C++. However, as became apparent in later stages of the coding segment of this project, there were certain language-based disadvantages that I couldn't find ways around with in C++. I'll give you some places I got stuck to show why I eventually decided to switch the language I coded this project in to Python. 

\begin{enumerate}
  \item JSON parsing is not natively available in C++. For that reason, I had to use an external library (JSONCpp \cite{jsoncpp}) and had to interface that with the rest of my files. This, whilst possible, took up a lot of space in the project directory as I manually included these files in my project folder\footnote{There were probably better ways to accomplish the same task in hindsight, but nonetheless, it would have been quite arduous either way.} as I didn't really want to go through the hassle of installing  a C++ package on my system that I would likely only be using for one project. This proved to be a problem when it came to storing elements as I started getting MLE errors. I tried to remedy this by eliminating many different data files that I wasn't directly using, but nonetheless, the error persisted. Eventually, this motivated me to try to switch to a language in which JSON parsing capabilities would be easier to achieve (and be less space-consuming). Among those, Python stood out to me because of its versatility and so I felt Python was a good candidate to switch over to,
  \item I wanted this project to be easily accessible and executable so it could remain accessible to as many people as possible. For those reasons, as Python is more accessible than C++ (both in terms of its popularity and executing programs), I felt Python was a better choice. 
\end{enumerate}

I switched over to Python based on the reasons I gave above. None of the logic itself really changed (as expected) and it was just a syntactical conversion from C++ to Python.